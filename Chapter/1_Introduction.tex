\documentclass[../Main.tex]{subfiles}
\begin{document}
\section{Motivation}
\label{section:1.1}
The rapid development of blockchain ecosystems has created a landscape of many independent networks, each operating with its own assets, smart contract environments, and liquidity pools. Although this diversity has contributed to innovation, it has also introduced a high level of fragmentation that complicates how users interact with digital value. Managing assets across chains remains cumbersome, and transferring liquidity often requires complex operational steps or reliance on intermediaries that weaken the trustless nature of decentralized finance.

Stablecoins, which have become a fundamental component of the digital economy, are still limited by their dependence on single-chain infrastructures or centralized issuers. These restrictions prevent stablecoins from functioning as a truly universal medium of exchange. Users who hold volatile assets on one network cannot easily unlock value on another without passing through bridging systems that may introduce delays, additional fees, and security risks. As a result, the current environment reduces capital efficiency and constrains the usefulness of decentralized financial applications.

Solving this fragmentation is vital to the long-term evolution of decentralized finance. A system that enables unified cross-chain asset management would not only improve how users mint or redeem stable assets but could also serve as a foundation for advanced applications such as multi-chain lending, derivatives, asset management tools, and liquidity optimization frameworks. A reliable mechanism for issuing and managing stable digital assets across chains would help unlock a more coherent, interoperable, and economically efficient blockchain ecosystem.\section{Objectives and scope of the graduation thesis}
\label{section:1.2}
In recent years, several models have attempted to address the challenge of creating stable digital assets that operate across diverse blockchain networks. Custodial stablecoins offer strong usability but depend on centralized entities for asset backing. Over-collateralized decentralized models such as MakerDAO rely on a single-chain design that is difficult to extend to multiple networks without complex bridging layers. Algorithmic stabilization mechanisms explore endogenous supply control but often fail under extreme market conditions. Meanwhile, solutions that rely on existing cross-chain bridges encounter liquidity fragmentation, inconsistent state synchronization, and heightened security risks.

These approaches reveal persistent limitations: user positions cannot be maintained in a unified state across chains, collateral management requires fragmented infrastructure, and expansion to new networks often relies on external systems that reduce security and reliability.

In response to these challenges, this thesis focuses on designing an architecture that maintains a consistent, global collateralized debt position that can be accessed and updated from multiple blockchain networks. The aim is to propose and implement a protocol in which users can lock collateral on one chain while minting stable assets on any supported chain, with the canonical state stored on Solana. The thesis scope includes constructing a multi-chain wallet abstraction, developing a secure verification and message-passing mechanism, enabling updates to user positions originating from external chains, and establishing a stablecoin model capable of minting and burning natively across networks. By addressing existing constraints, the proposed design moves toward a more interoperable multi-chain system for stable asset issuance without dependence on centralized bridging counterparts.
\section{Tentative solution}
\label{section:1.3}
To approach the problem defined above, this thesis adopts a design centered on Solana as the authoritative settlement layer, combined with message verification components and smart contracts deployed across EVM networks that serve as user entry points for submitting signed requests. Solana maintains all universal collateralized debt positions through deterministic Program Derived Addresses, which define a unified wallet structure for each user regardless of the number of chains or external wallets they interact with.

User actions such as providing collateral, minting stable assets, repaying obligations, or redeeming collateral are represented as structured requests that include identifiers for the originating chain and sequence information for replay protection. These requests are verified on EVM-side contracts, then observed and relayed by off-chain guardians or backend processes to Solana. The Solana gateway contract validates the messages and forwards them to the main protocol contract, which processes state transitions in accordance with the CDP logic.

The essential contributions of this thesis are the design of a unified multi-chain CDP architecture, the introduction of a secure method for processing cross-chain messages, and the implementation of a mint-and-burn mechanism for stable assets that operates on multiple chains while maintaining a single source of truth on Solana. The expected outcome is a functional prototype that demonstrates secure and synchronized cross-chain stablecoin issuance, along with consistent solvency and reliable state management.
\section{Thesis organization}
\label{section:1.4}
The structure of this thesis is designed to guide the reader through the full development process of the proposed multi-chain stablecoin system, beginning with foundational motivations and ending with practical implementation and evaluation. Each chapter plays a specific role in shaping the final solution and collectively ensures that the research narrative progresses logically from problem identification to system deployment.

Chapter 2 presents a comprehensive requirement survey and analysis. It begins by examining the current situation and the technological context in which multi-chain stablecoin solutions operate. This includes an exploration of existing systems, user needs, and the challenges posed by fragmented blockchain environments. The chapter then introduces the functional overview of the proposed system, incorporating both general and detailed use case diagrams that illustrate how users interact with the system across different contexts. The business processes are also discussed to provide a clear understanding of how information flows through the system. Following this, the chapter offers detailed descriptions of key use cases and concludes with an analysis of the system’s non-functional requirements, such as security, performance, scalability, and reliability, which set the baseline for design constraints in later chapters.

Chapter 3 focuses on the methodology adopted in conducting the research and building the system. It outlines the reasoning behind selecting specific technologies, development frameworks, and verification models. The chapter explains the methodological steps taken to ensure scientific rigor, including how the multi-chain architecture was evaluated, how cross-chain communication assumptions were validated, and how the Solana-centered design philosophy influences the broader system. By defining the methodological foundation, the chapter ensures that subsequent design and implementation decisions are grounded in a consistent and justified approach.

Chapter 4 provides an extensive discussion of the system’s design, implementation, and evaluation. It begins with the architecture design, explaining the rationale behind software architecture choices and presenting a detailed overview of the system’s components and their interactions. The discussion continues with package-level and module-level design, followed by a thorough specification of the user interface layout, the layered backend design, and the on-chain data structures used in both Solana and EVM environments. The chapter also describes the database schema where applicable, the tools and libraries used throughout development, and the milestones achieved during the building phase. It then illustrates the major functional flows of the system and explains how they were tested to ensure correctness, security, and performance. The chapter concludes with a discussion of the deployment process, including how the system is prepared for real-world execution across multiple blockchain networks.

Chapter 5 highlights the complete solution and the key contributions of the thesis. It synthesizes the work from previous chapters into a coherent model that demonstrates how the system solves the multi-chain stablecoin problem in a unified and scalable way. This chapter also emphasizes the technical and conceptual innovations introduced by the thesis, including unified wallet abstraction, cross-chain request verification, canonical CDP storage on Solana, and multi-chain mint–burn logic for stablecoin issuance. The contributions are contextualized within the broader landscape to clearly show how the proposed solution advances the state of the art.

Finally, Chapter 6 concludes the thesis by summarizing the main results achieved and discussing their implications for blockchain interoperability and decentralized finance. It reflects on the strengths and limitations of the system, and it provides several directions for future work, such as supporting additional chains, improving guardian decentralization, enhancing message throughput, or integrating advanced risk management mechanisms. By outlining these potential extensions, the thesis opens a path for continued development and academic exploration.\end{document}