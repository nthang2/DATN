\documentclass[../Main.tex]{subfiles}
\begin{document}

\begin{center}
    \Large{\textbf{ABSTRACT}}\\
\end{center}
\vspace{1cm}

As blockchain ecosystems continue to expand, the lack of a unified mechanism for maintaining credit, collateralization, and stable value across heterogeneous networks has become a significant limitation for decentralized finance. Existing stablecoin models are predominantly single-chain or rely on centralized bridging infrastructures, resulting in fragmented liquidity, duplicated state, and increased security risks. Although several approaches have attempted to enable cross-chain value transfer, they often introduce trust assumptions, inconsistent state tracking, or high operational complexity. To address these challenges, this thesis adopts a multi-chain architecture built on a Solana-centered Collateralized Debt Position (CDP) system combined with decentralized message relaying from external chains. This approach ensures that all credit, collateral, and risk management logic is executed on a high-performance chain while allowing users to interact from any supported network.

The primary contributions of this thesis include a unified CDP state machine on Solana, a robust cross-chain request verification protocol, a multi-chain mint, burn mechanism without liquidity fragmentation. Experimental validation demonstrates reliable nonce synchronization, secure request execution, and consistent system solvency across chains. This work provides a practical and extensible foundation for interoperable multi-chain stablecoin protocols.
\begin{flushright}
\begin{tabular}{@{}c@{}}
Student\\
\textit{(Signature and full name)}
\end{tabular}
\end{flushright}
\end{document}